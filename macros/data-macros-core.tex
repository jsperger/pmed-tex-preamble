%% data-macros-core.tex
%% Author: John Sperger and contributors

%% Defines the core (i.e. no shorthand commands) commands related to
%% the data in a paper. 


\usepackage{amsmath}
\usepackage{amssymb}


%%%%%%%%%%%%%%%%%%%%%%%%%%%%%%%%%%%%%%%%%%%%%%%%%%%%%%%%%%%%%%%%%%%%%%%%%%%%%%%%
% Data
% 
%%%%%%%%%%%%%%%%%%%%%%%%%%%%%%%%%%%%%%%%%%%%%%%%%%%%%%%%%%%%%%%%%%%%%%%%%%%%%%%%


% Define a new command to create a set of macros for 
% data. Using the input letter it defines:
% \(pre)space Space = letter in mathcal
% \(pre)rv Random variable = uppercase letter
% \(pre)obs Observed instantiation = lowercase letter
% \(pre)rvvec Vector of RVs = bold face capital letter
% \(pre)obsvec Vector of observations = bold face lowercase letter

\newcommand{\DefineDataMacros}[2]{%
  \expandafter\newcommand\csname #1letter\endcsname{#2} % Define the basic letter
  \expandafter\newcommand\csname #1space\endcsname{\mathcal{\csname #1letter\endcsname}} % Define the space notation
  \expandafter\newcommand\csname #1rv\endcsname{\MakeUppercase{\csname #1letter\endcsname}} % Define the random variable (uppercase)
  \expandafter\newcommand\csname #1obs\endcsname{\MakeLowercase{\csname #1letter\endcsname}} % Define the observation (lowercase)
  \expandafter\newcommand\csname #1rvvec\endcsname{\boldsymbol{\csname #1rv\endcsname}} % Define the vector of random variables
  \expandafter\newcommand\csname #1obsvec\endcsname{\boldsymbol{\csname #1obs\endcsname}} % Define the vector of observations
}

% Define data-related core macros
\DefineDataMacros{covar}{X} % covariates
\DefineDataMacros{arm}{A} % actions/arms
\DefineDataMacros{resp}{Y} % response



%%%%%%%%%%%%%%%%%%%%%%%%%%%%%%%%%%%%%%%%%%%%%%%%%%%%%%%%%%%%%%%%%%%%%%%%%%%%%%%%
% Indices
% 
%%%%%%%%%%%%%%%%%%%%%%%%%%%%%%%%%%%%%%%%%%%%%%%%%%%%%%%%%%%%%%%%%%%%%%%%%%%%%%%%

% Define a new command to create a set of macros for 
% data. Using the input letter it defines:
% \(pre)index Index = lowercase letter
% \(pre)maxindex Maximum index = uppercase letter

\newif\ifindicesdefined % Declare a new conditional
\indicesdefinedfalse   % Set the conditional to false initially

\newcommand{\DefineIndex}[1]{%
  \ifindicesdefined
    \errmessage{Indices already defined!} % Raise an error if indices have already been defined
  \else
    \ifx#1basic
      \DefineIndexSet{ind}{n}
    \fi
    \ifx#1long
      \DefineIndexSet{cluster}{i}
      \DefineIndexSet{ind}{j}
      \DefineIndexSet{time}{k}
    \fi
    \ifx#1sw
      \DefineIndexSet{cluster}{i}
      \DefineIndexSet{time}{j}
      \DefineIndexSet{ind}{k}
    \fi
    \ifx#1mab
      \DefineIndexSet{arm}{k}
      \DefineIndexSet{time}{t}
    \fi
    \ifx#1cb
      \DefineIndexSet{ind}{i}
      \DefineIndexSet{arm}{k}
      \DefineIndexSet{time}{t}
    \fi
    \indicesdefinedtrue % Set the conditional to true to indicate that indices have been defined
  \fi
}

% Helper command to define individual macros
\newcommand{\DefineIndexSet}[2]{%
  \expandafter\newcommand\csname #1index\endcsname{\MakeLowercase{#2}} % Define the basic index (lowercase letter)
  \expandafter\newcommand\csname #1maxindex\endcsname{\MakeUppercase{#2}} % Define the maximum index (uppercase letter)
}

\DefineIndexSet{'basic'}
