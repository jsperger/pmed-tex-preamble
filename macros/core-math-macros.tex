%% data-macros-core.tex
%% Author: John Sperger and contributors

%% Defines the core (i.e. no shorthand commands) commands related to
%% the data in a paper.


\usepackage{amsmath}
\usepackage{amssymb}


%%%%%%%%%%%%%%%%%%%%%%%%%%%%%%%%%%%%%%%%%%%%%%%%%%%%%%%%%%%%%%%%%%%%%%%%%%%%%%%%
% Special Symbols
%
%%%%%%%%%%%%%%%%%%%%%%%%%%%%%%%%%%%%%%%%%%%%%%%%%%%%%%%%%%%%%%%%%%%%%%%%%%%%%%%%

\newcommand{\dv}{\, \mathrm{d} } % for writing integrals
\newcommand{\rootn}{\sqrt{n}}

% Linear Algebra
\newcommand{\idmat}{\mathtt{I}} % identity matrix
\newcommand{\trans}{\mathtt{T}} % transpose

%%%%%%%%%%%%%%%%%%%%%%%%%%%%%%%%%%%%%%%%%%%%%%%%%%%%%%%%%%%%%%%%%%%%%%%%%%%%%%%%
% Operators
%
%%%%%%%%%%%%%%%%%%%%%%%%%%%%%%%%%%%%%%%%%%%%%%%%%%%%%%%%%%%%%%%%%%%%%%%%%%%%%%%%
% Fundamentals
\DeclareMathOperator*{\argmax}{argmax} % argmax - in display mode subscripts go directly under the argmax
\DeclareMathOperator*{\argmin}{argmin} % argmin- in display mode subscripts go directly under the argmin
\DeclareMathOperator{\indfun}{{\boldsymbol{1}}} % indicator function
\newcommand{\norm}[1]{\left\lVert#1\right\rVert} % norm. Argument to the command goes inside the double lines ||arg||
\DeclareMathOperator{\sign}{sign} % sign function

% Linear Algebra
\DeclareMathOperator{\blkdiag}{blkdiag}
\DeclareMathOperator{\diag}{diag}
\DeclareMathOperator{\vect}{vec} % vec operator
