%% data-macros-core.tex
%% Author: John Sperger and contributors

%% Defines the core (i.e. no shorthand commands) commands related to
%% the data in a paper.


\usepackage{amsmath}
\usepackage{amssymb}


%%%%%%%%%%%%%%%%%%%%%%%%%%%%%%%%%%%%%%%%%%%%%%%%%%%%%%%%%%%%%%%%%%%%%%%%%%%%%%%%
% Data
%
%%%%%%%%%%%%%%%%%%%%%%%%%%%%%%%%%%%%%%%%%%%%%%%%%%%%%%%%%%%%%%%%%%%%%%%%%%%%%%%%


%%%%%%%%%%%%%%%%%%%%%%%%%%%%%%%%%%%%%%%%%%%%%%%%%%%%%%%%%%%%%%%%%%%%%%%%%%%%%%%%
% Operators
%
%%%%%%%%%%%%%%%%%%%%%%%%%%%%%%%%%%%%%%%%%%%%%%%%%%%%%%%%%%%%%%%%%%%%%%%%%%%%%%%%
% Fundamentals
\DeclareMathOperator*{\argmax}{argmax} % argmax - in display mode subscripts go directly under the argmax
\DeclareMathOperator*{\argmin}{argmin} % argmin- in display mode subscripts go directly under the argmin
\DeclareMathOperator{\indfun}{{\boldsymbol{1}}} % indicator function
\newcommand{\norm}[1]{\left\lVert#1\right\rVert} % norm. Argument to the command goes inside the double lines ||arg||
\DeclareMathOperator{\sign}{sign} % sign function

% Linear Algebra
\DeclareMathOperator{\blkdiag}{blkdiag}
\DeclareMathOperator{\diag}{diag}
\DeclareMathOperator{\vect}{vec} % vec operator

% Statistics
\DeclareMathOperator{\avar}{{\mathrm{Avar}}} % asymptotic variance
\DeclareMathOperator{\cov}{{\mathrm{cov}}} % covariance
\DeclareMathOperator{\expt}{{\rm I\kern-.3em E}} % expectation
\DeclareMathOperator{\KL}{\rm KL} % KL-divergence
\DeclareMathOperator{\mse}{{\mathrm{MSE}}} % mean squared error
\DeclareMathOperator{\prob}{{\mathrm{Pr}}} % Probability
\DeclareMathOperator{\var}{{\mathrm{Var}}} % variance
