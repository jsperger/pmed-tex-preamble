\documentclass{article}

% Required packages
\usepackage{etoolbox} 
\usepackage{amsmath}
\usepackage{amssymb}

% Commands you provided go here
%%%%%%%%%%%%%%%%%%%%%%%%%%%%%%%%%%%%%%%%%%%%%%%%%%%%%%%%%%%%%%%%%%%%%%%%%%%%%%%%
% Required packages
% 
%%%%%%%%%%%%%%%%%%%%%%%%%%%%%%%%%%%%%%%%%%%%%%%%%%%%%%%%%%%%%%%%%%%%%%%%%%%%%%%%
\usepackage{amsmath}
\usepackage{amssymb}
\usepackage{etoolbox} % required for the programatic definition of alphabetic math symbols

%%%%%%%%%%%%%%%%%%%%%%%%%%%%%%%%%%%%%%%%%%%%%%%%%%%%%%%%%%%%%%%%%%%%%%%%%%%%%%%%
% Notes
% 
%%%%%%%%%%%%%%%%%%%%%%%%%%%%%%%%%%%%%%%%%%%%%%%%%%%%%%%%%%%%%%%%%%%%%%%%%%%%%%%%

% Commands for operators and symbols should use all lower case letters

% Commands changing the font for a character should use lower case letters for the portion
% of the command representing the style and a character matching the desired output 
% i.e. an upper-case letter if the output will be an upper-case letter. Examples:
% \mbbR for \mathbb{R} not \mbbr for \mathbb{R}


%%%%%%%%%%%%%%%%%%%%%%%%%%%%%%%%%%%%%%%%%%%%%%%%%%%%%%%%%%%%%%%%%%%%%%%%%%%%%%%%
% Precision Medicine Related Quantities
% 
%%%%%%%%%%%%%%%%%%%%%%%%%%%%%%%%%%%%%%%%%%%%%%%%%%%%%%%%%%%%%%%%%%%%%%%%%%%%%%%%


% Data, Estimators, Etc. 
%%%%%%%%%%%%%%%%%%%%%%%%%%%%%%%%%%%%%%%%%%%%%%%%%%%%%%%%%%%%%%%%%%%%%%%%%%%%%%%%

%%% Sets and Spaces related
% Fundamentals
\newcommand{\covarspace}{\mathcal{X}} % covariate space
\newcommand{\armspace}{\mathcal{A}} % action space
\newcommand{\outspace}{\mathcal{Y}} % outcome space
\newcommand{\histspace}{\mathcal{H}} % history space

% Shorthand
\newcommand{\cspc}{\covarspace}% covariate space
\newcommand{\aspc}{\armspace} % action space
\newcommand{\ospc}{\outspace} % outcome space
\newcommand{\yspc}{\outspace} % outcome space
\newcommand{\hspc}{\histspace} % history space

%%% Data
% Fundamentals
\newcommand{\covardist}{\mathcal{D}} % Covariate distribution
\newcommand{\covarmat}{X} % Covariate 
\newcommand{\obs}{x} % observed 
\newcommand{\hist}{H} % history
\newcommand{\resp}{Y} % response

% Shorthand
\newcommand{\histt}{H_t} % history up to time t shorthand
\newcommand{\histn}{H_n} % history up to n observations shorthand

%%% Policy and Parameter related symbols
% Fundamentals
\newcommand{\pol}{\pi} % generic policy
\newcommand{\optpol}{\pi^{*}} % optimal policy
\newcommand{\armparam}{\beta} % Parameter vector for a generic arm
\newcommand{\polhat}{\widehat{\pol}} % estimated policy
\newcommand{\armhat}{\widehat{\armparam}} % esimated arm parameter vector

% Shorthand
\newcommand{\polhatn}{\polhat_{n}} % shorthand for estimated policy based on a sample size of n
\newcommand{\armparamk}{\armparam_{k}} % Parameter vector for arm k


% Functions
%%%%%%%%%%%%%%%%%%%%%%%%%%%%%%%%%%%%%%%%%%%%%%%%%%%%%%%%%%%%%%%%%%%%%%%%%%%%%%%%

%%% Precision Medicine Specific
% Fundamentals
\DeclareMathOperator{\val}{{\mathcal{V} }} % value function
\DeclareMathOperator{\BR}{BR} % Bayes Regret
\DeclareMathOperator{\R}{\rm R} % Shortened form for regret
\DeclareMathOperator{\Reg}{\rm Reg} % Regret
\DeclareMathOperator{\Regpop}{\rm Reg^{pop}} %population regret
\newcommand{\reghat}{\widehat{\Reg}} % Estimated Regret
\DeclareMathOperator{\valhat}{{\widehat{\val}}} % estimated value function

% Shorthand
\DeclareMathOperator{\valhatn}{{\widehat{\val}}_{n}} % estimated value function estimated on a sample of size n

%%%%%%%%%%%%%%%%%%%%%%%%%%%%%%%%%%%%%%%%%%%%%%%%%%%%%%%%%%%%%%%%%%%%%%%%%%%%%%%%
% Generally Useful Commands
% 
%%%%%%%%%%%%%%%%%%%%%%%%%%%%%%%%%%%%%%%%%%%%%%%%%%%%%%%%%%%%%%%%%%%%%%%%%%%%%%%%

%%% Operators
% Fundamentals
\DeclareMathOperator{\indfun}{{\boldsymbol{1}}} %indicator function
\DeclareMathOperator{\sign}{sign} %sign function
\DeclareMathOperator{\prob}{{\mathrm{Pr}}} % Probability
\DeclareMathOperator{\expt}{{\rm I\kern-.3em E}} % expectation
\DeclareMathOperator{\diag}{diag} 
\DeclareMathOperator{\blkdiag}{blkdiag}
\DeclareMathOperator{\KL}{\rm KL} % KL-divergence
\DeclareMathOperator{\avar}{{\mathrm{Avar}}} % asymptotic variance
\DeclareMathOperator{\vect}{vec} %vec operator
\DeclareMathOperator*{\argmax}{argmax} % argmax - in display mode subscripts go directly under the argmax
\DeclareMathOperator*{\argmin}{argmin} % argmin- in display mode subscripts go directly under the argmin
\DeclareMathOperator{\var}{{\mathrm{Var}}} % variance
\DeclareMathOperator{\cov}{{\mathrm{cov}}} % covariance
\DeclareMathOperator{\mse}{{\mathrm{MSE}}} % mean squared error
\newcommand{\norm}[1]{\left\lVert#1\right\rVert} % norm. Argument to the command goes inside the double lines ||arg||  

% Shorthand

\DeclareMathOperator{\E}{\expt} %expectation shorthand

%%% Alphabetic shorthand for Mathcal and Mathbb for every capital letter in the English alphabet

\def\do#1{%
  \expandafter\gdef\csname mc#1\endcsname{{\mathcal{#1}}}%
  \expandafter\gdef\csname mbb#1\endcsname{{\mathbb{#1}}}%
  \expandafter\gdef\csname msf#1\endcsname{{\mathsf{#1}}}%
}

%The \docsvlist command is provided by the etoolbox package and is 
% responsible for iterating through a comma-separated list of values and 
% applying a specified command to each value.
% 'do csv list'
\docsvlist{A,B,C,D,E,F,G,H,I,J,K,L,M,N,O,P,Q,R,S,T,U,V,W,X,Y,Z}


%%% Command abbreviations
\def\bs{\boldsymbol} 
\def\l{\left}
\def\r{\right}   

\begin{document}

% Now we test each command
The covariate space is $\covarspace$ or $\cspc$.
\begin{verbatim}
\covarspace or \cspc: represents the covariate space.
\end{verbatim}

The action space is $\armspace$ or $\aspc$.
\begin{verbatim}
\armspace or \aspc: represents the action space.
\end{verbatim}

The outcome space is $\outspace$ or $\ospc$ or $\yspc$.
\begin{verbatim}
\outspace or \ospc or \yspc: represents the outcome space.
\end{verbatim}

The history space is $\histspace$ or $\hspc$.
\begin{verbatim}
\histspace or \hspc: represents the history space.
\end{verbatim}

The covariate distribution is $\covardist$.
\begin{verbatim}
\covardist: represents the covariate distribution.
\end{verbatim}

The covariate matrix is $\covarmat$.
\begin{verbatim}
\covarmat: represents the covariate matrix.
\end{verbatim}

The observed variable is $\obs$.
\begin{verbatim}
\obs: represents the observed variable.
\end{verbatim}

The history is $\hist$.
\begin{verbatim}
\hist: represents the history.
\end{verbatim}

The response is $\resp$.
\begin{verbatim}
\resp: represents the response.
\end{verbatim}

The history up to time $t$ is $\histt$.
\begin{verbatim}
\histt: represents the history up to time t.
\end{verbatim}

The history up to $n$ observations is $\histn$.
\begin{verbatim}
\histn: represents the history up to n observations.
\end{verbatim}

The generic policy is $\pol$.
\begin{verbatim}
\pol: represents the generic policy.
\end{verbatim}

The optimal policy is $\optpol$.
\begin{verbatim}
\optpol: represents the optimal policy.
\end{verbatim}

The parameter vector for a generic arm is $\armparam$.
\begin{verbatim}
\armparam: represents the parameter vector for a generic arm.
\end{verbatim}

The estimated policy is $\polhat$.
\begin{verbatim}
\polhat: represents the estimated policy.
\end{verbatim}

The estimated arm parameter vector is $\armhat$.
\begin{verbatim}
\armhat: represents the estimated arm parameter vector.
\end{verbatim}

The estimated policy based on a sample size of $n$ is $\polhatn$.
\begin{verbatim}
\polhatn: represents the estimated policy based on a sample size of n.
\end{verbatim}

The parameter vector for arm $k$ is $\armparamk$.
\begin{verbatim}
\armparamk: represents the parameter vector for arm k.
\end{verbatim}

The value function is $\val$.
\begin{verbatim}
\val: represents the value function.
\end{verbatim}

The Bayes regret is $\BR$.
\begin{verbatim}
\BR: represents the Bayes regret.
\end{verbatim}

The regret is $\R$ or $\Reg$.
\begin{verbatim}
\R or \Reg: represents the regret.
\end{verbatim}

The population regret is $\Regpop$.
\begin{verbatim}
\Regpop: represents the population regret.
\end{verbatim}

The estimated regret is $\reghat$.
\begin{verbatim}
\reghat: represents the estimated regret.
\end{verbatim}

The estimated value function is $\valhat$.
\begin{verbatim}
\valhat: represents the estimated value function.
\end{verbatim}

The estimated value function based on a sample size of $n$ is $\valhatn$.
\begin{verbatim}
\valhatn: represents the estimated value function based on a sample size of n.
\end{verbatim}

The indicator function is $\indfun$.
\begin{verbatim}
\indfun: represents the indicator function.
\end{verbatim}

The sign function is $\sign$.
\begin{verbatim}
\sign: represents the sign function.
\end{verbatim}

The probability is $\prob$.
\begin{verbatim}
\prob: represents the probability.
\end{verbatim}

The expectation is $\expt$ or $\E$.
\begin{verbatim}
\expt or \E: represents the expectation.
\end{verbatim}

The diagonal is $\diag$.
\begin{verbatim}
\diag: represents the diagonal of a matrix.
\end{verbatim}

The block diagonal is $\blkdiag$.
\begin{verbatim}
\blkdiag: represents the block diagonal of a matrix.
\end{verbatim}

The KL-divergence is $\KL$.
\begin{verbatim}
\KL: represents the KL-divergence.
\end{verbatim}

The asymptotic variance is $\avar$.
\begin{verbatim}
\avar: represents the asymptotic variance.
\end{verbatim}

The vec operator is $\vect$.
\begin{verbatim}
\vect: represents the vec operator.
\end{verbatim}

The argmax is $\argmax$.
\begin{verbatim}
\argmax: represents the argmax function.
\end{verbatim}

The argmin is $\argmin$.
\begin{verbatim}
\argmin: represents the argmin function.
\end{verbatim}

The variance is $\var$.
\begin{verbatim}
\var: represents the variance.
\end{verbatim}

The covariance is $\cov$.
\begin{verbatim}
\cov: represents the covariance.
\end{verbatim}

The mean squared error is $\mse$.
\begin{verbatim}
\mse: represents the mean squared error.
\end{verbatim}

The norm of a vector is $\norm{v}$.
\begin{verbatim}
\norm{v}: represents the norm of a vector v.
\end{verbatim}

And finally, the boldsymbol command can be used as follows: $\bs{a}$
\begin{verbatim}
\bs{a}: makes the letter a bold.
\end{verbatim}

\end{document}
